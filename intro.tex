\section{Introduction}

\index{generate}

\subsection{Background}

Program synthesis is the problem of generating a program from a certain specification, often expressed as a logical constraint. The problem we consider is one of Programming-by-Example synthesis, where instead of a formal specification, we attempt to generate a program using input-output pairs. This is useful in situations where we are interested in discovering exact, possibly complex patterns in a piece of data. This method has strong limitations, as you might expect, but has the advantages that it can work even when datasets are far too small to use statistical methods, that we make very few assumptions about the data (beyond the fact that it has some computable pattern), and that once we generate a program, we can examine and completely understand its behaviour.

Some real and potential uses of this kind of technology include:

\begin{itemize}
  \item "Flash Fill"~\cite{gulwani2017program}: This is the technology behind Microsoft Excel's autocomplete feature.It is what allows the software to detect \& extend a user's actions. For example, given a 
  
\end{itemize}

\subsection{Motivation \& Goals}

\subsection{Structure}

\subsection{Technological Decisions}

